\clearpage
\phantomsection
\chapter*{前言}
\addcontentsline{toc}{chapter}{前言}
得益于国内(曾经)对盗版软件的放纵,许多人通过一个叫做 Ballance
的游戏接触到了一个定价不菲%
\footnote{\$5000,\underline{\href{https://web.archive.org/web/20151016140349/http://www.gamasutra.com/view/feature/131377/product_review_virtools_dev_20.php}{见此}}}%
的 3D 应用开发环境 "Virtools" 。甚至此环境的一个中文社区在其简介中都提到了%
「众多 Virtools 爱好者就是从这款游戏开始认识研究 Virtools 软件的,%
并且有数个自做的版本在网上流传。」%
\footnote{\underline{\url{http://tieba.baidu.com/p/2256977468}},十楼}。%
但是多年来,涉及到 Virtools 作为 3D 应用开发环境所拥有的编程功能者%
少之又少。最主要的原因当然是因为关注度最高的制图方向并不需要相关的功能。
\par
不过从 2018 年起,事情发生了转变。年初, 2jjy 和我合作解密了全部 Ballance
的游戏脚本。2018 年底,我发现了在适用于原版游戏的地图中注入脚本的方式。%
这两个发现,在国内的 Ballance 玩家里吹起了一股学习 Virtools 脚本的新风。%
奈何当下已难觅得 Virtools 脚本的中文资料,遂起了编写本指南的念头。
\par
Virtools 脚本所用的图形化编程方式向来有直观易用的名声,%
甚至获得了「使美工也能学会编程」的美誉%
\footnote{链接同注2,四楼}。相信读者在投入些许时间后,一定能对 Virtools
脚本有相当程度的了解。同时本书在最后还将为有相关基础的读者简单地介绍
Virtools SDK 以及相关的高级内容。
\par
需要注意的是,本书的假想读者是对 Virtools 脚本感兴趣的 Ballance 玩家。%
由于在本书写作的时候,Virtools 已经宣布废弃近十年,
如果您单纯是想学习 Virtools 的相关技术,强烈建议您转向其他更新的开发技术。%
游戏有情怀可言,但时间对过时的技术来讲是毫不留情的。
\clearpage
